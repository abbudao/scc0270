\documentclass[11pt,a4paper]{report}
\usepackage[portuguese]{babel}
\usepackage[utf8]{inputenc}
\usepackage{amsmath}
\usepackage{amsfonts}
\usepackage{tikz}
\usetikzlibrary{calc}
\usepackage{float}
\usepackage{graphicx}
\def\layersep{3.5cm}
\begin{document}
\begin{titlepage}
\centering
\includegraphics[width=0.33\textwidth]{usp}\par\vspace{1cm}
{\scshape\LARGE Universidade de São Paulo\par}
  \vspace{1cm}
  {\scshape\Large SCE5809 - Redes Neurais\par}
  \vspace{1.5cm}
  {\huge\bfseries Exercício em Sala No.1\par}
  \vspace{2cm}
  {\Large\itshape Pedro Morello Abbud \par}
  \vspace{1cm}
  Número USP 8058718
  \vfill
  Disciplina ministrada por\par
 Profa. Titular Roseli Ap. Francelin Romero 

\vfill 
% Bottom of the page
{\large \today\par}
\end{titlepage}

\section{Enunciado}
Fazer uma iteração do algoritmo \emph{BackPropagation} (ida e volta) para treinar uma rede multi-camadas a aprender a função XOR.
\begin{figure}[htpb]
  \centering
\begin{tikzpicture}[shorten >=1pt,->,draw=black!50, node distance=\layersep]
    \tikzstyle{every pin edge}=[<-,shorten <=1pt]
    \tikzstyle{neuron}=[circle,fill=black!25,minimum size=17pt,inner sep=0pt]
    \tikzstyle{input neuron}=[neuron, fill=green!50];
    \tikzstyle{output neuron}=[neuron, fill=red!50];
    \tikzstyle{hidden neuron}=[neuron, fill=blue!50];
    \tikzstyle{annot} = [text width=4em, text centered]

    % Draw the input layer nodes
    \foreach \name / \y in {1,...,2}
    % This is the same as writing \foreach \name / \y in {1/1,2/2,3/3,4/4}
        \node[input neuron, pin=left:$x_\y$] (I-\name) at (0,-\y) {};

    % Draw the hidden layer nodes
    \foreach \name / \y in {1,...,2}
        \path[yshift=0cm]
            node[hidden neuron] (H-\name) at (\layersep,-\y cm) {};

    % Draw the output layer node
            \node[output neuron,pin={[pin edge={->}]right:Output}, right of= H-1] at ($(H-1)!.5!(H-2)$) (O) {};

    % Connect every node in the input layer with every node in the
    % hidden layer.
    \foreach \source in {1,...,2}
    {
        \foreach \dest in {1,...,2}
        {
          \ifodd \dest 
          \draw (I-\source) -- (H-\dest) node [pos=0.25,left,  sloped,text width=1em] (w\source\dest) {$w_{\source\dest}$};
          \else
          \draw (I-\source) -- (H-\dest) node [pos=0.75,left,  sloped,text width=1em] (w\source\dest) {$w_{\source\dest}$};
          \fi
%            \path (I-\source) edge (H-\dest);
%            \node (w\source\dest) at ($(I-\source)!.5!(H-\dest)$) {$w_\source\dest$};
        }
    }
    % Connect every node in the hidden layer with the output layer
    \foreach \source in {1,...,2}
        \path (H-\source) edge (O);
    % Annotate the layers
    \node[annot,above of=H-1, node distance=1cm] (hl) {Hidden layer};
    \node[annot,left of=hl] {Input layer};
    \node[annot,right of=hl] {Output layer};
\end{tikzpicture}

  \caption{}
  \label{fig:}
\end{figure}

\section{Resolução}
\end{document}
